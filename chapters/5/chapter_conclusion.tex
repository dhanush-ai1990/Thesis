\startchapter{Conclusions and Future Work}
\label{concl}

Distributional Semantic Models (DS) or word vectors that are based on text corpora are vital for various Natural Language Processing (NLP) related tasks. Because of the immense applications of these models, there is a sheer need to perform evaluation and comparison of different DS models. BrainBench is a system designed to evaluate and benchmark word vectors using brain data published by Xu et al. in 2016~\cite{BrainBench2016}. In this thesis, we publish the second iteration of BrainBench incorporating two new Italian brain datasets collected using fMRI and EEG technology. Doing so, we improved the coverage of the number of words supported by BrainBench from 60 to 190. 

Another important add-on to the BrainBench suite is the ability to evaluate word-vectors against anatomical regions of interest (ROIs) in human brain. This add-on subsequently could help the computational linguistics community studying language and visual representation in the human brain using word vectors. We also conducted experiments to investigate the performance of abstract concepts in word-vectors against BrainBench. The results of this study indicate that abstract concepts show lower correlation to brain data as compared to the concrete concepts. We also provide evidence to show that DS models exhibit correlation to EEG brain signals. EEG data is much cheaper and convenient to collect as compared to fMRI and MEG. Our results with the EEG dataset, therefore, should encourage the scientific community to build more EEG based tools to evaluate and benchmark word vectors.

Although BrainBench is a robust and effective tool to evaluate word-vectors, it only has a coverage of 190 concepts even with the contributions by this thesis. Moreover, it still does not contain concepts related to other parts of speech such as adjectives, verbs, pronouns etc. The coverage of abstract nouns is only 15\% of the total words. Therefore, we propose that more brain datasets need to be added to BrainBench suite to improve the coverage of concepts and effectively making it more feasible for practical evaluation of word vectors.

Another significant contribution of this thesis is the study of semantic representation in Convolutional Neural Networks (CNN). CNNs are a computational model that has become the state of the art technology in object recognition from images. However, there is an apparent lack of understanding on why various CNN architectures perform better than the other.  We used the same word vectors evaluated by BrainBench to study CNNs. In short, we asked the question: Do a CNN learn semantics?

Our results indicate that CNNs do indeed learn semantics. The semantic information in these networks grows in an upward trend from the first convolutional layer and peaks at the layer before the classification layer. We demonstrated that our methodology could explain complex architectures such as VGG16, ResNet50, and Inception-v3 by studying the semantic representations through the hidden layers of these networks. In the case of misclassifications by a CNN,  we observed that the semantic information required to make the correct classification decision does exist in the intermediate layers of some networks, even if the classification layer makes wrong predictions. We hope our methodology and results could potentially pave the way for improved design and debugging of CNN.

In this thesis, we focused our study on only three CNNs: VGG16, ResNet50 and Inception-v3. There are many diverse CNNs released everywhere that could be studied with our methodology. Another exciting area of interest among the computer vision community is the study of adversarial attacks on CNNs. The techniques proposed in this work could be modified to detect adversarial attacks on CNNs. Since the hidden layers of CNNs correlate with the semantic information in word-vectors, we could potentially train CNNs using both images and word-vectors jointly to learn a shared semantic representation. Such methods might improve the generalization of these networks and could make them more robust to adversarial attacks. 
